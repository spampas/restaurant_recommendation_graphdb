\chapter{Application's modules}\label{ch:modules}
In this chapter we present briefly the modules that compose our application. 
The application is Client-Server, the client has a graphical user interface and
the server is a stateful multi-threaded server which use a Neo4j database for
data persistency.

	\section{Client}\label{sec:client}
	The client application has been developed using JavaFX for the graphical
	user interface. The Views used are organized in Panels, which groups
	functionalities for searching Users and Restaurant, a form for User 
	Preferences and Restaurants creation and modification. Furthermore the
	Users with administrator functionalities will have an administration
	panel where it is possible to add Cities and Cuisines. Further
	information on the Client module can be found in \ref{ch:usermanual}
	\begin{figure}[!h]
		\includegraphics[width=\textwidth]{client}
		\caption*{\textbf{Figure~\ref{fig:client}}}
		\captionlistentry{}
		\label{fig:client}
	\end{figure}

	\section{Server}\label{sec:server}
	The server is a multithreaded stateful server. The multithreading has
	been achieved using a reusable thread pool. 
	For each user connected the state is maintained on the server
	for all the connection time.
	The connection with the database use bolt protocol, a Neo4j specific
	protocol, which being native is faster and lighter then HTTP based 
	connection. The Data Access Layer has been designed using a Object Graph
	Mapping library, thus allowing us to easily perform CRUD operation on
	the DB and to implement complex functionalities using Cypher queries. 
	To avoid memory exhausting problems the Driver for the database use a
	pool of sessions that can be reused among different threads, this has
	some consequences from the point of view of the consistency between
	memory and database, since some operations may be deferred or done using
	cached data. Thus to avoid inconsistency problem every time a request is
	finished we clear and close the session so that the driver can recreate
	a new one, so that the memory can be freed from cached data and loaded 
	at the next request. In this way  we can be sure that the data on which 
	we work is completely consistent with the database and that we have in
	memory only the data needed for the operation.
	\begin{figure}[!h]
		\includegraphics[width=\textwidth]{server}
		\caption*{\textbf{Figure~\ref{fig:server}}}
		\captionlistentry{}
		\label{fig:server}
	\end{figure}

	\section{Common classes}\label{sec:common}
	The common classes are serializable classes that are shared between the
	application modules. These classes are used for communication purposes.


	\begin{figure}[!h]
		\includegraphics[width=\textwidth]{common}
		\caption*{\textbf{Figure~\ref{fig:common}}}
		\captionlistentry{}
		\label{fig:common}
	\end{figure}
